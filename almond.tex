% Options for packages loaded elsewhere
\PassOptionsToPackage{unicode}{hyperref}
\PassOptionsToPackage{hyphens}{url}
%
\documentclass[
]{article}
\usepackage{lmodern}
\usepackage{amssymb,amsmath}
\usepackage{ifxetex,ifluatex}
\ifnum 0\ifxetex 1\fi\ifluatex 1\fi=0 % if pdftex
  \usepackage[T1]{fontenc}
  \usepackage[utf8]{inputenc}
  \usepackage{textcomp} % provide euro and other symbols
\else % if luatex or xetex
  \usepackage{unicode-math}
  \defaultfontfeatures{Scale=MatchLowercase}
  \defaultfontfeatures[\rmfamily]{Ligatures=TeX,Scale=1}
\fi
% Use upquote if available, for straight quotes in verbatim environments
\IfFileExists{upquote.sty}{\usepackage{upquote}}{}
\IfFileExists{microtype.sty}{% use microtype if available
  \usepackage[]{microtype}
  \UseMicrotypeSet[protrusion]{basicmath} % disable protrusion for tt fonts
}{}
\makeatletter
\@ifundefined{KOMAClassName}{% if non-KOMA class
  \IfFileExists{parskip.sty}{%
    \usepackage{parskip}
  }{% else
    \setlength{\parindent}{0pt}
    \setlength{\parskip}{6pt plus 2pt minus 1pt}}
}{% if KOMA class
  \KOMAoptions{parskip=half}}
\makeatother
\usepackage{xcolor}
\IfFileExists{xurl.sty}{\usepackage{xurl}}{} % add URL line breaks if available
\IfFileExists{bookmark.sty}{\usepackage{bookmark}}{\usepackage{hyperref}}
\hypersetup{
  pdftitle={Almond Model},
  pdfauthor={Anna Abelman, Julia Dagum, Jaleise Hall},
  hidelinks,
  pdfcreator={LaTeX via pandoc}}
\urlstyle{same} % disable monospaced font for URLs
\usepackage[margin=1in]{geometry}
\usepackage{color}
\usepackage{fancyvrb}
\newcommand{\VerbBar}{|}
\newcommand{\VERB}{\Verb[commandchars=\\\{\}]}
\DefineVerbatimEnvironment{Highlighting}{Verbatim}{commandchars=\\\{\}}
% Add ',fontsize=\small' for more characters per line
\usepackage{framed}
\definecolor{shadecolor}{RGB}{248,248,248}
\newenvironment{Shaded}{\begin{snugshade}}{\end{snugshade}}
\newcommand{\AlertTok}[1]{\textcolor[rgb]{0.94,0.16,0.16}{#1}}
\newcommand{\AnnotationTok}[1]{\textcolor[rgb]{0.56,0.35,0.01}{\textbf{\textit{#1}}}}
\newcommand{\AttributeTok}[1]{\textcolor[rgb]{0.77,0.63,0.00}{#1}}
\newcommand{\BaseNTok}[1]{\textcolor[rgb]{0.00,0.00,0.81}{#1}}
\newcommand{\BuiltInTok}[1]{#1}
\newcommand{\CharTok}[1]{\textcolor[rgb]{0.31,0.60,0.02}{#1}}
\newcommand{\CommentTok}[1]{\textcolor[rgb]{0.56,0.35,0.01}{\textit{#1}}}
\newcommand{\CommentVarTok}[1]{\textcolor[rgb]{0.56,0.35,0.01}{\textbf{\textit{#1}}}}
\newcommand{\ConstantTok}[1]{\textcolor[rgb]{0.00,0.00,0.00}{#1}}
\newcommand{\ControlFlowTok}[1]{\textcolor[rgb]{0.13,0.29,0.53}{\textbf{#1}}}
\newcommand{\DataTypeTok}[1]{\textcolor[rgb]{0.13,0.29,0.53}{#1}}
\newcommand{\DecValTok}[1]{\textcolor[rgb]{0.00,0.00,0.81}{#1}}
\newcommand{\DocumentationTok}[1]{\textcolor[rgb]{0.56,0.35,0.01}{\textbf{\textit{#1}}}}
\newcommand{\ErrorTok}[1]{\textcolor[rgb]{0.64,0.00,0.00}{\textbf{#1}}}
\newcommand{\ExtensionTok}[1]{#1}
\newcommand{\FloatTok}[1]{\textcolor[rgb]{0.00,0.00,0.81}{#1}}
\newcommand{\FunctionTok}[1]{\textcolor[rgb]{0.00,0.00,0.00}{#1}}
\newcommand{\ImportTok}[1]{#1}
\newcommand{\InformationTok}[1]{\textcolor[rgb]{0.56,0.35,0.01}{\textbf{\textit{#1}}}}
\newcommand{\KeywordTok}[1]{\textcolor[rgb]{0.13,0.29,0.53}{\textbf{#1}}}
\newcommand{\NormalTok}[1]{#1}
\newcommand{\OperatorTok}[1]{\textcolor[rgb]{0.81,0.36,0.00}{\textbf{#1}}}
\newcommand{\OtherTok}[1]{\textcolor[rgb]{0.56,0.35,0.01}{#1}}
\newcommand{\PreprocessorTok}[1]{\textcolor[rgb]{0.56,0.35,0.01}{\textit{#1}}}
\newcommand{\RegionMarkerTok}[1]{#1}
\newcommand{\SpecialCharTok}[1]{\textcolor[rgb]{0.00,0.00,0.00}{#1}}
\newcommand{\SpecialStringTok}[1]{\textcolor[rgb]{0.31,0.60,0.02}{#1}}
\newcommand{\StringTok}[1]{\textcolor[rgb]{0.31,0.60,0.02}{#1}}
\newcommand{\VariableTok}[1]{\textcolor[rgb]{0.00,0.00,0.00}{#1}}
\newcommand{\VerbatimStringTok}[1]{\textcolor[rgb]{0.31,0.60,0.02}{#1}}
\newcommand{\WarningTok}[1]{\textcolor[rgb]{0.56,0.35,0.01}{\textbf{\textit{#1}}}}
\usepackage{graphicx,grffile}
\makeatletter
\def\maxwidth{\ifdim\Gin@nat@width>\linewidth\linewidth\else\Gin@nat@width\fi}
\def\maxheight{\ifdim\Gin@nat@height>\textheight\textheight\else\Gin@nat@height\fi}
\makeatother
% Scale images if necessary, so that they will not overflow the page
% margins by default, and it is still possible to overwrite the defaults
% using explicit options in \includegraphics[width, height, ...]{}
\setkeys{Gin}{width=\maxwidth,height=\maxheight,keepaspectratio}
% Set default figure placement to htbp
\makeatletter
\def\fps@figure{htbp}
\makeatother
\setlength{\emergencystretch}{3em} % prevent overfull lines
\providecommand{\tightlist}{%
  \setlength{\itemsep}{0pt}\setlength{\parskip}{0pt}}
\setcounter{secnumdepth}{-\maxdimen} % remove section numbering

\title{Almond Model}
\author{Anna Abelman, Julia Dagum, Jaleise Hall}
\date{4/8/2021}

\begin{document}
\maketitle

\hypertarget{summary}{%
\subsubsection{Summary}\label{summary}}

After developing an R function to represent the California almond yield
anomaly regression model from
\href{https://gauchospace.ucsb.edu/courses/mod/resource/view.php?id=6782707}{\emph{Lobell
2006}}, the yield anomalies for all years 1989 - 2010 were calculated.
We found that the year in which there was the most extreme anomaly was
in 1995 where the yield anomaly was nearly 2000 ton
acre\textsuperscript{-1}. There were other much smaller spikes in 1997,
2005, and 2008. With the full production of almond crop being 6 years,
we may see such a high spike in 1995 due to the crop reaching it's full
maturity. This reasoning requires the assumption that the almond crops
studies in \emph{Lobell 2006} were planted at the start of the study
period.

\begin{Shaded}
\begin{Highlighting}[]
\CommentTok{#read in the clim data}
\NormalTok{clim <-}\StringTok{ }\KeywordTok{read.table}\NormalTok{(}\StringTok{"clim.txt"}\NormalTok{, }\DataTypeTok{sep=}\StringTok{" "}\NormalTok{, }\DataTypeTok{header=}\NormalTok{T)}

\KeywordTok{source}\NormalTok{(}\StringTok{"almond_model.R"}\NormalTok{)}
\end{Highlighting}
\end{Shaded}

\begin{Shaded}
\begin{Highlighting}[]
\CommentTok{#create temperature subset for February}
\NormalTok{temperature <-}\StringTok{ }\NormalTok{clim }\OperatorTok\StringTok{ }
\StringTok{  }\KeywordTok{filter}\NormalTok{(month }\OperatorTok{==}\StringTok{ "2"}\NormalTok{) }\OperatorTok\StringTok{ }
\StringTok{  }\KeywordTok{group_by}\NormalTok{(year) }\OperatorTok\StringTok{ }
\StringTok{  }\KeywordTok{summarize}\NormalTok{(}
    \DataTypeTok{avg =} \KeywordTok{mean}\NormalTok{(tmin_c)}
\NormalTok{  ) }
\end{Highlighting}
\end{Shaded}

\begin{Shaded}
\begin{Highlighting}[]
\CommentTok{#create precipitation subset for January}
\NormalTok{rain <-}\StringTok{ }\NormalTok{clim }\OperatorTok\StringTok{ }
\StringTok{  }\KeywordTok{filter}\NormalTok{(month }\OperatorTok{==}\StringTok{ "1"}\NormalTok{) }\OperatorTok\StringTok{ }
\StringTok{  }\KeywordTok{group_by}\NormalTok{(year) }\OperatorTok\StringTok{ }
\StringTok{  }\KeywordTok{summarize}\NormalTok{(}
    \DataTypeTok{sum =} \KeywordTok{sum}\NormalTok{(precip)}
\NormalTok{  )}

\CommentTok{#combine the temperature and precipitation data}
\NormalTok{df <-}\StringTok{ }\KeywordTok{data.frame}\NormalTok{(temperature, rain) }\OperatorTok\StringTok{ }
\StringTok{  }\KeywordTok{select}\NormalTok{(}\OperatorTok{-}\NormalTok{year}\FloatTok{.1}\NormalTok{)}
\end{Highlighting}
\end{Shaded}

\begin{Shaded}
\begin{Highlighting}[]
\CommentTok{#test the model on the clim data}
\KeywordTok{almond_model}\NormalTok{(clim)}
\end{Highlighting}
\end{Shaded}

\begin{verbatim}
##  [1]   -0.3552237    9.2906757   68.9130633   15.4280698   20.2083803
##  [6]    2.4820009 1919.9811511    3.5818399  329.6938750   27.8636956
## [11]   -0.1436364    9.5999883  159.5119587    0.2450914   -0.2585997
## [16]   -0.2367722  656.3724121   18.6324135   20.2007396  576.2821943
## [21]    0.7367438  153.7655092
\end{verbatim}

\begin{Shaded}
\begin{Highlighting}[]
\CommentTok{#turn the model output into a data frame}
\NormalTok{almond_yield_anomaly <-}\StringTok{ }\KeywordTok{data.frame}\NormalTok{(}\DataTypeTok{year =}\NormalTok{ rain}\OperatorTok{$}\NormalTok{year, }\DataTypeTok{anomaly =} \KeywordTok{almond_model}\NormalTok{(clim)) }\OperatorTok\StringTok{ }
\StringTok{  }\KeywordTok{mutate}\NormalTok{(}\DataTypeTok{year =}\NormalTok{ lubridate}\OperatorTok{::}\KeywordTok{ymd}\NormalTok{(year, }\DataTypeTok{truncated =}\NormalTok{ 2L))}
  
\CommentTok{#plot the anomalies}
\KeywordTok{ggplot}\NormalTok{(}\DataTypeTok{data =}\NormalTok{ almond_yield_anomaly, }\KeywordTok{aes}\NormalTok{(}\DataTypeTok{y =}\NormalTok{ anomaly, }\DataTypeTok{x =}\NormalTok{ year)) }\OperatorTok{+}
\StringTok{  }\KeywordTok{geom_line}\NormalTok{() }\OperatorTok{+}
\StringTok{  }\KeywordTok{scale_x_date}\NormalTok{(}\DataTypeTok{date_breaks =} \StringTok{"1 year"}\NormalTok{,}
               \DataTypeTok{date_labels =} \StringTok{"%Y"}\NormalTok{,}
               \DataTypeTok{limits =} \KeywordTok{as.Date}\NormalTok{(}\KeywordTok{c}\NormalTok{(}\StringTok{"1989-01-01"}\NormalTok{,}\StringTok{"2010-01-01"}\NormalTok{))) }\OperatorTok{+}
\StringTok{  }\KeywordTok{labs}\NormalTok{(}\DataTypeTok{x =} \StringTok{"Year"}\NormalTok{,}
       \DataTypeTok{y =} \KeywordTok{expression}\NormalTok{(}\StringTok{"Anomaly (ton"} \OperatorTok{~}\NormalTok{acre}\OperatorTok{^-}\DecValTok{1}\OperatorTok{~}\StringTok{ ")"}\NormalTok{),}
       \DataTypeTok{title =} \StringTok{"Annual Almond Yield Anomaly (1989 - 2010)"}\NormalTok{) }\OperatorTok{+}
\StringTok{  }\KeywordTok{theme_minimal}\NormalTok{() }\OperatorTok{+}
\StringTok{  }\KeywordTok{theme}\NormalTok{(}\DataTypeTok{panel.grid.minor.x =} \KeywordTok{element_blank}\NormalTok{(), }
        \DataTypeTok{plot.title =} \KeywordTok{element_text}\NormalTok{(}\DataTypeTok{hjust =} \FloatTok{0.5}\NormalTok{),}
        \DataTypeTok{axis.text.x =} \KeywordTok{element_text}\NormalTok{(}\DataTypeTok{angle =} \DecValTok{60}\NormalTok{, }\DataTypeTok{vjust =} \DecValTok{1}\NormalTok{, }\DataTypeTok{hjust =} \DecValTok{1}\NormalTok{))}
\end{Highlighting}
\end{Shaded}

\begin{figure}
\centering
\includegraphics{almond_files/figure-latex/unnamed-chunk-5-1.pdf}
\caption{Almond yield anomaly in California for all years 1989 to 2010.
The anomaly is calculated using the regression model: Y =
-0.015T\textsubscript{n,2} -
0.0046T\textsuperscript{2}\textsubscript{n,2} - 0.07P\textsubscript{1} +
0.0043P\textsuperscript{2}\textsubscript{1} + 0.28}
\end{figure}

\end{document}
